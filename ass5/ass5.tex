\documentclass[12pt]{article}
\usepackage{amsmath}
\usepackage{graphicx}
\usepackage{hyperref}
\usepackage[latin1]{inputenc}

\title{Virtual Reality\\*Assignment 5}
\author{Le Anh Phuong\\* Dang Thi Hoang Yen}
\date{16/12/18}

\begin{document}
\maketitle

\section*{Exercise 5.1}
\subsection*{a)}

Apply rotation to navigation node.\\
Let's create navigation2 node as a child of intersection node, such that:
$$n2.WT = n.WT $$
$$=> i.T * n2.T = n.T$$
$$=> n2.T = inv(i.T) * n.T
$$
Apply rotation to navigation2 node, new navigation2 local transform is:
$$n2.T = R * n2.T = R * inv(i.T) * n.T$$
New transform matrix of navigation node is:
$$n.T = i.T * n2.T$$
$$=>n.T = i.T * R * inv(i.T) * n.T$$
\subsection*{b)}

Transform matrix of pointer in intersection coordinate system is:
$$A = inv(i.T) * n.T * p.T$$
A consists of rotation and translation.\\
Distance from pointer to intersection would be calculated based on the translation matrix of A.\\
$$
A.translate=\begin{bmatrix}
    1 & 0 & 0 & tx \\
    0 & 1 & 0 & ty \\
    0 & 0 & 1 & tz \\
    0 & 0 & 0 & 1
  \end{bmatrix}
$$
$$Z = distance = sqrt(tx^2 + ty^2 + tz^2)$$
\\\\Let's attach intersection2 node as a child of intersection node, which represents new coordinate system and pointer2 node as a child of intersection2 node, which represents pointer in this new coordinate system.
We need to calculate transform matrix of intersection2 node.
$$M = i2.T$$
$$p2.WT = p.WT$$
$$=> i.T * i2.T * p2.T = n.T * p.T$$
As requirement, local transform of pointer2 represents -z-translation of the pointer to the intersection.
$$=> P = p2.T = \begin{bmatrix}
    1 & 0 & 0 & 0 \\
    0 & 1 & 0 & 0 \\
    0 & 0 & 1 & -Z \\
    0 & 0 & 0 & 1
  \end{bmatrix}$$
Putting it together,
$$i.T * M * P = n.T * p.T$$
$$=> M = inv(i.T) * n.T * p.T * inv(P)$$

\subsection*{c)}
delta is the transform matrix of current frame pointer in relation to last frame pointer.

\subsection*{d)}
Model transform of wheel1\_geom 
$$w1^W = cycle\_trans.T * wheel1\_trans.T * wheel1\_geom$$
View transform of wheel1\_geom 
$$w1^{S} = inv(s.T) * inv(n.T) * w1^W$$
Model View transform of wheel1\_geom 
$$w1^S = inv(s.T) * inv(n.T) * cycle\_trans.T * wheel1\_trans.T * wheel1\_geom$$

\section*{Exercise 5.2}
First, scale the screen, then rotate and translate.\\
Transformations matrix for front screen 
$$s1.T = trans(0,1,-2) * scale(4,2,1)$$
Transformations matrix for left screen 
$$s2.T = trans(-2,1,0) * rot(90, y) * scale(4,2,1)$$
Transformations matrix for right screen 
$$s3.T = trans(2,1,0) * rot(-90, y) * scale(4,2,1)$$

\section*{Exercise 5.3}
As stated in slide 15 in stereoscopic viewing lecture, maximal positive disparity is equal to eye distance, hence, for this set up the maximal positive disparity is 0.065 m and 41 pixel (2560 / 400 * 6.5).\\\\
From the maximum disparity in pixels, we derive the number of depth levels is 41 levels .

\section*{Exercise 5.4}
Refer to slide 18 in viewing setup lecture, we can summerize the parameters given as followed:
$$d = 2m$$
$$o_x = 0.7m$$
$$o_y = 1.75m$$
$$w = 4m$$
$$h = 2m$$
$$z_{near} = 0.05m$$
$$z_{far} = 100m$$
We calculate left and right boundaries of viewing window in the $z_{near}$ plane as followed:
$$\frac{l}{o_x-w/2} = \frac{z_{near}}{d}$$
$$=>\frac{l}{0.7-2} = \frac{0.05}{2}$$
$$=>l = -0.0325$$\\
$$\frac{r}{o_x+w/2} = \frac{z_{near}}{d}$$
$$=>\frac{r}{0.7+2} = \frac{0.05}{2}$$
$$=>r = 0.0675$$
Similarly, we calculate left and right boundaries of viewing window in the $z_{far}$ plane as followed:
$$\frac{l}{-o_x-w/2} = \frac{z_{far}}{d}$$
$$=>\frac{l}{-0.7-2} = \frac{100}{2}$$
$$=>l = -135$$\\
$$\frac{r}{-o_x+w/2} = \frac{z_{far}}{d}$$
$$=>\frac{r}{-0.7+2} = \frac{100}{2}$$
$$=>r = 65$$
\end{document}